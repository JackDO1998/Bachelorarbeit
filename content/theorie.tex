\section{Theorie}
\label{sec:Theorie}

\subsection{Entstehung von Synchrotronstrahlung}
\label{sec:Synchrotronstrahlung}
Synchrotronstrahlung bezeichnet ein kontinuirliches Strahlungspektrum das vom Infraroten über das 
sichtbare und ultraviolette Licht bis in den Bereich der harten Röntgenstrahlung reicht. Sie entsteht, 
wie bereits erwähnt, durch die beschleunigung von elektrisch geladenen Teilchen, im speziellen Fall des
DELTA Speicherrings durch die Beschleunigung von Elektronen. Um die zur Beschleunigung verwendete 
Richtungsänderung zu bewirken werden Magnetfelder verwendet. Elektrisch geladene Teilchen in einem 
Magnetfeld sind der Lorentzkraft ausgezetzt. Sie wird beschrieben durch:

\begin{equation*}
    \vec{F_L} = e \cdot \vec{v} \times \vec{B} = \dot{\vec{p}}
\end{equation*}

Die Kraft wirkt also auf ein mit der Ladung $e$ belegtes Teilchen das mit der Geschwindigkeit $\vec{v}$
das Magnetfeld $\vec{B}$ durchuqert. Der Biegeradius $R$ lässt sich durch gleichstzen der Lorentzkraft 
mit der Zentripetalkraft $\vec{F_Z}$ ermitteln. 

\begin{equation*}
    \vec{F_Z} = m \frac{\vec{v^2}}{R}
\end{equation*}

\begin{equation*}
    \vec{F_Z} = \vec{F_L} \\\
    \Rightarrow m \frac{\vec{v}^2}{R} = e \cdot \vec{v} \times \vec{B}\\\\
    \Leftrightarrow R = \frac{\vec{v}^2 m}{e \cdot \lvert \vec{v} \times \vec{B} \rvert}
\end{equation*}

Da sich die Elektronen mit nahezu Lichtgeschwindigkeit bewegen muss relativistisch gerechnet werden.
Es gilt also $m=m_0\gamma$ mit $m_0$ der Ruhemasse des Elektrons und $\gamma$ dem Lorentzfaktor.
Die Strahlungsleistung $P_S$ eines derart bewegten Elektrons wird beschrieben durch:

\begin{equation*}
    P_S = \frac{e^2c}{6\pi \epsilon_0}\frac{1}{(m_0c^2)^4}\frac{E^4}{R^2}
\end{equation*}

Hier meint $c$ die Vakuumlichtgeschwindigkeit, $\epsilon_0$ die Dielektrizitätskonstante 
und $E$ die Gesamtenergie des Teilchens. Das die Ruhemase $m_0$ in vierter Potenz eingeht,
erklärt die gute Eignung von Elektronen zur Synchrotronstrahlungserzeugung. Schwerere 
Teilchen wie Protonen oder auch \tau-Leptonen sind nicht nur schwerer zu beschaffen, sondern
müssten auch auf deutlich größere Energien beschleunigt werden um die gleiche Strahlungsleistung
zu erbringen. Pro Umlauf ergibt sich für ein Elektron die mittlere Abgestrahlte Energie $\Delta E$:

\begin{equation*}
    \Delta E = \frac{e^2}{3\epsilon_0(m_oc^2)^4}\frac{E^4}{R}
\end{equation*}

\subsubsection{Undulatorstrahlung}
\label{sec:Undulatorstrahlung}
Undulatoren zählen zu den sogenannten incertion devices, diese werden auf graden Strecken des 
Speicherrings untergebracht welche ohne diese Geräte einfache Driftstrecken wären. Ein Undulator besteht
aus einer Abfolge von abwechselnd gepolten Magneten. Bei den Magneten kann es sich sowohl um Dauer- als
auch normal- oder supraleitende Elektromagneten handeln. Durch die Periodische, abwechselnd gepolte 
Anordnung von Magneten beschreiben die durfliegenden Teilchen eine sinusförmige Bahn. Die Amplitude des 
Sinus ist dabei Abhängig von der Magnetischen Feldstärke und der Gesamtmasse der Teilchen. Eine 
characterristische Größe ist hier der sogenannte Undulatorparameter $K$, er ergibt sich über:

\begin{equation*}
    K \equiv \frac{\lambda_U e B_0}{2 \pi m_e c}
\end{equation*}

Dabei bezeichnet $\lambda_U$ die Undulatorperiode, also den longitudinalen Abstand zwischen zwei 
gleichartig gepolten Magneten. Anhand des Undulatorparameters lassen sich auch Undulatoren von Wigglern
unterscheiden. Für einen Undulator gilt $K \leq 1$ und für einen Wiggler $K > 1$. Bei einem Undulator 
interferieren die in jedem Bogen der sinusförmigen Bahn erzeugten Strahlungskegel miteinander was zu
einem Linienspektrum führt. Die Intensität der Synchrotronstrahlung ist hier um den Faktor $N_P^2$ 
erhöht, wobei $N_P$ die Anzahl der Magnetpole ist. Bei einem Wiggler erfolgt keine Interferenz und 
es entsteht ein kontinuirliches Spektrum dessen Synchrotronstrahlungsintensität proportional zu $N_P$ ist.
Diese Unterscheidung ist jedoch sehr theoretisch, in der Praxis treten mit zunehmender Flussdichte im 
Undulatorspektrum immer mehr ungradzahlige Harmonische der Undulatorline auf die dann zu einem 
kontinuirlichen Spektrum verschmelzen.




\subsection{Wahrscheinlichkeit für das Messen eines Photons}
\label{sec:Wahrscheinlichkeitsrechnung}

\subsection{Höhenkorrektur der Histogramme}
\label{sec:TheorieKorrektur}


