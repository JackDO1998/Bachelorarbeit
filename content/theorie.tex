\section{Theorie}
\label{sec:Theorie}
In diesem Kapitel soll kurz die nötige Theoretische Grundlage geschaffen werden.

\subsection{Entstehung von Synchrotronstrahlung}
\label{sec:Synchrotronstrahlung}
Synchrotronstrahlung bezeichnet ein kontinuirliches Strahlungspektrum das vom Infraroten über das 
sichtbare und ultraviolette Licht bis in den Bereich der harten Röntgenstrahlung reicht. Sie entsteht, 
wie bereits erwähnt, durch die beschleunigung von elektrisch geladenen Teilchen, im speziellen Fall des
DELTA Speicherrings durch die Beschleunigung von Elektronen. Um die zur Beschleunigung verwendete 
Richtungsänderung zu bewirken werden Magnetfelder verwendet. Elektrisch geladene Teilchen in einem 
Magnetfeld sind der Lorentzkraft ausgezetzt. Sie wird beschrieben durch:

\begin{equation*}
    \vec{F_L} = e \cdot \vec{v} \times \vec{B} = \dot{\vec{p}}
\end{equation*}

Die Kraft wirkt also auf ein mit der Ladung $e$ belegtes Teilchen das mit der Geschwindigkeit $\vec{v}$
das Magnetfeld $\vec{B}$ durchuqert. Der Biegeradius $R$ lässt sich durch gleichstzen der Lorentzkraft 
mit der Zentripetalkraft $\vec{F_Z}$ ermitteln. 

\begin{equation*}
    \vec{F_Z} = m \frac{\vec{v^2}}{R}
\end{equation*}

\begin{equation*}
    \vec{F_Z} = \vec{F_L} \\\
    \Rightarrow m \frac{\vec{v}^2}{R} = e \cdot \vec{v} \times \vec{B}\\\\
\end{equation*}

\begin{equation}
    \label{eq:Biegeradius}
    \Leftrightarrow R = \frac{\vec{v}^2 m}{e \cdot \lvert \vec{v} \times \vec{B} \rvert}
\end{equation}

Da sich die Elektronen mit nahezu Lichtgeschwindigkeit bewegen muss relativistisch gerechnet werden.
Es gilt also $m=m_0\gamma$ mit $m_0$ der Ruhemasse des Elektrons und $\gamma$ dem Lorentzfaktor.
Die Strahlungsleistung $P_S$ eines derart bewegten Elektrons wird beschrieben durch:

\begin{equation*}
    P_S = \frac{e^2c}{6\pi \epsilon_0}\frac{1}{(m_0c^2)^4}\frac{E^4}{R^2}
\end{equation*}

Hier meint $c$ die Vakuumlichtgeschwindigkeit, $\epsilon_0$ die Dielektrizitätskonstante 
und $E$ die Gesamtenergie des Teilchens. Das die Ruhemase $m_0$ in vierter Potenz eingeht,
erklärt die gute Eignung von Elektronen zur Synchrotronstrahlungserzeugung. Schwerere 
Teilchen wie Protonen oder auch \tau-Leptonen sind nicht nur schwerer zu beschaffen, sondern
müssten auch auf deutlich größere Energien beschleunigt werden um die gleiche Strahlungsleistung
zu erbringen. Pro Umlauf ergibt sich für ein Elektron die mittlere Abgestrahlte Energie $\Delta E$:

\begin{equation*}
    \Delta E = \frac{e^2}{3\epsilon_0(m_oc^2)^4}\frac{E^4}{R}
\end{equation*}

\subsubsection{Undulatorstrahlung}
\label{sec:Undulatorstrahlung}
Undulatoren zählen zu den sogenannten incertion devices, diese werden auf graden Strecken des 
Speicherrings untergebracht welche ohne diese Geräte einfache Driftstrecken wären. Ein Undulator besteht
aus einer Abfolge von abwechselnd gepolten Magneten. Bei den Magneten kann es sich sowohl um Dauer- als
auch normal- oder supraleitende Elektromagneten handeln. Durch die Periodische, abwechselnd gepolte 
Anordnung von Magneten beschreiben die durfliegenden Teilchen eine sinusförmige Bahn. Die Amplitude des 
Sinus ist dabei Abhängig von der Magnetischen Feldstärke und der Gesamtmasse der Teilchen. Eine 
characterristische Größe ist hier der sogenannte Undulatorparameter $K$, er ergibt sich über:

\begin{equation*}
    K \equiv \frac{\lambda_U e B_0}{2 \pi m_e c}
\end{equation*}

Dabei bezeichnet $\lambda_U$ die Undulatorperiode, also den longitudinalen Abstand zwischen zwei 
gleichartig gepolten Magneten. Anhand des Undulatorparameters lassen sich auch Undulatoren von Wigglern
unterscheiden. Für einen Undulator gilt $K \leq 1$ und für einen Wiggler $K > 1$. Bei einem Undulator 
interferieren die in jedem Bogen der sinusförmigen Bahn erzeugten Strahlungskegel miteinander was zu
einem Linienspektrum führt. Die Intensität der Synchrotronstrahlung ist hier um den Faktor $N_P^2$ 
erhöht, wobei $N_P$ die Anzahl der Magnetpole ist. Bei einem Wiggler erfolgt keine Interferenz und 
es entsteht ein kontinuirliches Spektrum dessen Synchrotronstrahlungsintensität proportional zu $N_P$ ist.
Diese Unterscheidung ist jedoch sehr theoretisch, in der Praxis treten mit zunehmender Flussdichte im 
Undulatorspektrum immer mehr ungradzahlige Harmonische der fundamentalen Undulatorline auf die dann zu 
einem kontinuirlichen Spektrum verschmelzen. Die Wellenlänge der fundamentalen Undulatorlinie kann 
geschrieben werden als:

\begin{equation*}
    \lambda = \lambda_U -\lambda_U \beta^*c \cos{\theta}
\end{equation*}

Dabei ist $\theta$ der Betrachterwinkel. Dieser ist von null verschieden falls der Beobachter sich 
nicht auf der Elektronenachse befindet. In den folgenden Experimenten ist der Beobachter jedoch immer
auf der Elektronenachse. Daher lässt sich vereinfacht schreiben:

\begin{equation*}
    \lambda = \lambda_U -\lambda_U \beta^*c
\end{equation*}

Um jedoch zu verhindern das der Beobachter vom Elektronenstrahl getroffen wird und dieser verloren geht
befindet sich zwischen Undulator und Beobachter noch ein Ablenkmagnet welcher Dipolstrahlung erzeugt.
Die Strahlung des Dipols überlagert sich nun mit unbekannter Phasenbeziehung mit der Strahlung des 
Undulators. Um die Relevanz der Dipolstrahlung im Vergleich zur Undulatorstarhlung abschätzen zu können
werden hier die Strahlungsleistungen im Bereich der Fundamentallinie verglichen.
Die Strahlungsleistung für dir fundamentallinie des Undulators beträgt in etwa:

\begin{equation*}
    P_{Undulator} [W] = 7,26\frac{E^2[GeV^2]I[A]K^2N_U}{\lambda_U[cm]} 
\end{equation*}

Die Gesamtstrahlungsleistung des Dipols lässt sich abschätzen über:

\begin{equation}
    P_{Total} = \frac{e\gamma^4}{3\epsilon_0R}I
\end{equation}

Da dies jedoch eben die Gesamtstrahlungsleistung im kompletten Spektrum ist und nicht die Leistung an der 
Undulatorlinie, muss diese Größe noch mit dem Integral der Spektralfunktion $S(x)$ an der entsprechenden 
Stelle gewichtet werden:

\begin{equation*}
    P_{Dipol} = P_{Total} \int_{a}^{b}S(x)dx = P_{Total} \int_{a}^{b} \frac{9\sqrt{3}}{8\pi}x\int_{x}^{\infty}K_{5/3}(u)du dx
\end{equation*}

mit $a = \frac{4\pi R}{3(\lambda-\lambda_{v}) \gamma^3}$ und $b = \frac{4\pi R}{3(\lambda+\lambda_{v}) \gamma^3}$ und $K_{5/3}(u)$
der modifizierten Besselfunktion zweiter Gattung zur Basis $5/3$.

\subsection{Wahrscheinlichkeit für das Messen eines Photons}
\label{sec:Wahrscheinlichkeitsrechnung}
Die Wahrscheinlichkeit das ein von einem Elektron in einem Bunch ausgestrahltes Photon
den Bandpassfilter passieren kann und nicht von einem Abschwächer absorbirt wird,
stzt sich aus den drei Einzelwahrscheinlichkeiten $P_{\textit{Abschwächer}}$,$P_{Bandpass}$ und $P_{𝐸𝑚𝑖𝑡𝑡𝑖𝑒𝑟𝑡}$
zusammen und wird aufgrund der Stichprobengröße von $N \approx 10^{11}$ als Normalverteilt
angenommen. Der Erwartungswert einer Normalverteilung ergibt sich aus: $\epsilon = N \times P_{Gesamt}$

\begin{equation*}
    \epsilon = N * P_{\textit{Abschwächer}} * P_{Bandpass} * P_{Emmitiert}
\end{equation*}

\subsubsection{Berechnung von \texorpdfstring{$P_{\textit{Abschwächer}}$}{TEXT}}
\label{sec:PAbschwaecher}
Die Wahrscheinlichkeit das ein Photon beliebiger Energie einen Abschwächer neutraler
Dichte durchqueren ergibt sich direkt aus dem Kehrwert seiner Dichte $\rho$. Für $n$ hintereinander stehende 
Abschwächer müssen diese Wahrscheinlichkeiten multipliziert werden. Für $n$ Abschwächer mit gleicher 
neutraler Dichte ergibt sich:

\begin{equation*}
    P_{\textit{Abschwächer}} = \frac{1}{\rho^n}
\end{equation*}

\subsubsection{Berechnung von \texorpdfstring{$P_{Bandpass}$}{TEXT}}
\label{sec:PBandpass}
Die Wahrscheinlichkeit das ein im Undulator abgestrahltes Photon die richtige Wellenlänge hat um den 
Bandpassfilter passieren zu können ergibt sich aus dem Integral über das Undulatorspektrum. Das 
Undulatorspektrum hat die folgende Form:



\subsection{Höhenkorrektur der Histogramme}
\label{sec:TheorieKorrektur}
Bedingt durch den messtechnischen Aufbau, kann immer nur eine bestimmte Anzahl an Photonen pro Messzyklus
gemessen werden. Dadurch ergebt es sich das es wahrscheinlicher ist ein Photon zu messen das kurz nach 
dem Start-Trigger ankommt als eines das lange nach dem Trigger ankommt. Denn je später das Photon ankommt
desto wahrscheinlicher ist es das zuvor schon so viele Photonen gemessen wurden das der Messspeicher 
für den entsprechenden Messzyklus bereits voll ist. Daher wurde ein Histogramm über viele Messzyklen 
aufgenommen und die Rohdaten aufgetragen. Da sich der Strahlstrom über die Messzeit praktisch nicht 
verändert hat, müsste der erste Balken des Histogramms ebenso hoch sein wie der 193., der 385. und der 
577.. Es ist jedoch ein scheinbar exponentieller Abfall zu erkennen. Daher wurden die Daten logarithmiert 
und durch die jeweils vier zusammengehörenden Balkenhöhen eine Ausgleichsgrade gelegt. Anschließend werden
Mittelwerte gebildet und die Logarithmierung rückgängig gemacht. Mit der entsandenen Funktion kann dann 
eine Höhenkorrektur durchgeführt werden.  



