\section{Theorie}
\label{sec:Theorie}

\subsection{Entstehung von Synchrotronstrahlung}
\label{sec:Synchrotronstrahlung}
Synchrotronstrahlung bezeichnet ein kontinuirliches Strahlungspektrum das vom Infraroten über das 
sichtbare und ultraviolette Licht bis in den Bereich der harten Röntgenstrahlung reicht. Sie entsteht, 
wie bereits erwähnt, durch die beschleunigung von elektrisch geladenen Teilchen, im speziellen Fall des
DELTA Speicherrings durch die Beschleunigung von Elektronen. Um die zur Beschleunigung verwendete 
Richtungsänderung zu bewirken werden Magnetfelder verwendet. Elektrisch geladene Teilchen in einem 
Magnetfeld sind der Lorentzkraft ausgezetzt. Sie wird beschrieben durch:

\begin{equation*}
    \vec{F_L} = e \cdot \vec{v} \times \vec{B} = \dot{\vec{p}}
\end{equation*}

Die Kraft wirkt also auf ein mit der Ladung $e$ belegtes Teilchen das mit der Geschwindigkeit $\vec{v}$
das Magnetfeld $\vec{B}$ durchuqert. Der Biegeradius $R$ lässt sich durch gleichstzen der Lorentzkraft 
mit der Zentripetalkraft $\vec{F_Z}$ ermitteln. 

\begin{equation*}
    \vec{F_Z} = m \frac{\vec{v^2}}{R}
\end{equation*}

\begin{equation*}
    \vec{F_Z} = \vec{F_L} \\\
    \Rightarrow m \frac{\vec{v}^2}{R} = e \cdot \vec{v} \times \vec{B}\\\\
    \Leftrightarrow R = \frac{\vec{v}^2 m}{e \cdot \lvert \vec{v} \times \vec{B} \rvert}
\end{equation*}

Da sich die Elektronen mit nahezu Lichtgeschwindigkeit bewegen muss relativistisch gerechnet werden.
Es gilt also $m=m_0\gamma$ mit $m_0$ der Ruhemasse des Elektrons und $\gamma$ dem Lorentzfaktor.
Die Strahlungsleistung $P_S$ eines derart bewegten Elektrons wird beschrieben durch:

\begin{equation*}
    P_S = \frac{e^2c}{6\pi \epsilon_0}\frac{1}{(m_0c^2)^4}\frac{E^4}{R^2}
\end{equation*}

Hier meint $c$ die Vakuumlichtgeschwindigkeit, $\epsilon_0$ die Dielektrizitätskonstante 
und $E$ die Gesamtenergie des Teilchens. Das die Ruhemase $m_0$ in vierter Potenz eingeht,
erklärt die gute Eignung von Elektronen zur Synchrotronstrahlungserzeugung. Schwerere 
Teilchen wie Protonen oder auch \tau-Leptonen sind nicht nur schwerer zu beschaffen, sondern
müssten auch auf deutlich größere Energien beschleunigt werden um die gleiche Strahlungsleistung
zu erbringen. Pro Umlauf ergibt sich für ein Elektron die mittlere Abgestrahlte Energie $\Delta E$:

\begin{equation*}
    \Delta E = \frac{e^2}{3\epsilon_0(m_oc^2)^4}\frac{E^4}{R}
\end{equation*}







\subsection{Wahrscheinlichkeit für das Messen eines Photons}
\label{sec:Wahrscheinlichkeitsrechnung}

\subsection{Höhenkorrektur der Histogramme}
\label{sec:TheorieKorrektur}


