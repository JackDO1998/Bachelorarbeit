\section{Durchführung}
\label{sec:Durchfuehrung}

\subsection{Auflösung des TDC bestimmen}
\label{sec:AufloesungTDC}

\subsection{Probleme bei der Messung mittels TDC}
\label{sec:ProblemeTDC}

\subsubsection{Geschwindigkeit des Delaygenerators}
\label{sec:Delaygenerator}
Damit der Versuchsaufbau wie ausgedacht funktioniert, müsste der Delaygenerator sein Delay alle 
$\SI{384}{\nano\second}$ um $\SI{2}{\nano\second}$ verlängern. Also zunächst $\SI{0}{\nano\second}$,
dann $\SI{2}{\nano\second}$, dann $\SI{4}{\nano\second}$ und so weiter bis $\SI{384}{\nano\second}$
um dann wieder bei $\SI{0}{\nano\second}$ zu beginnen. Wie sich jedoch zeigte ist der verwendete
Delaygenerator dazu nicht in der Lage. Dies könnte sowohl an der Konstuktion des Delaygenerators an sich 
liegen oder an dem verwendeten Aufbau in welchem der Delaygenerator mit dem Netzwerkprotokoll VXI-11
über das allgemein verwendete Python Script gesteuert wurde. Daher wurde der Delaygenerator aus dem 
Aufbau entfernt und es wird versucht die Daten so gut wie möglich nachträglich mathematisch zu korrigieren.


\subsubsection{Geschwindigkeit des TDC}
\label{sec:GeschwindigkeitTDC}


\subsection{Messungen mit dem Oszilloskop}
\label{sec:MessungenOszilloskop}

\subsection{Messungen mit dem Time Tagger}
\label{sec:MessungenTT}
